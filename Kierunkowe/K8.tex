\subsection{Weryfikacja modelowa z zastosowaniem logiki temporalnej}

Logika temporalna jest to rozszerzenie logiki tradycyjnej o symbole określające upływ czasu. Weryfikacja modelowa pozwala odpowiedzieć na pytanie czy formalny model funkcji systemu dany jako automat spełnia własności, zdefiniowane za pomocą formuł logiki temporalnej. Główne zastosowania to:


\begin{itemize}
	\item zarządzanie temporalnymi bazami danych,
	\item opis systemów współbieżnych,
	\item opis systemów reagujących na bodźce,
	\item określanie własności systemów,
	\item automatyczna weryfikacja programów – matematyczne udowodnienie ich poprawności.
\end{itemize}

Najważniejsze zadania weryfikacji:

\begin{itemize}
	\item osiągalność (pożądany stan zostanie \textbf{w końcu} osiągnięty)
	\item bezpieczeństwo (gwarancja, iż stan nieprawidłowy \textbf{nigdy} nie zostanie osiągnięty)
\end{itemize}

\subsubsection{Typy logiki temporalnej}

Wyróżniamy 3 typy logiki temporalnej:

\begin{itemize}
	\item \textbf{LTL} (\textit{Linear Temporal Logic}) – z liniową strukturą czasu,
	\item \textbf{CTL} (\textit{Computation Tree Logic}) – rozszerzenie logiki LTL o rozgałęzione warianty upływu czasu). Czas jest dyskretny, może się rozgałęziać (ale od pewnego momentu, wcześniej jest liniowy) oraz lewostronnie skończony (w przyszłości nieskończony). Zastosowanie : w działających współbieżnie programach , w systemach gdzie istnieje wiele wariantów upływu czasu,
	\item \textbf{Real Time CTL} - dalsze rozwiniecie logiki temporalnej, pozwala na weryfikacje systemów czasu rzeczywistego, gdzie dana operacja nie tylko musi być wykonana, ale też są na nią ograniczone ramy czasowe.
\end{itemize}

\subsubsection{Zalety i wady}

Zalety:

\begin{itemize}
	\item pełna automatyzacja (po utworzeniu wymagań i zdefiniowaniu ograniczeń wystarczy uruchomić weryfikację),
	\item łatwe dowodzenie nieprawidłowości przez znalezienie kontrprzykładu.
\end{itemize}

Wady

\begin{itemize}
	\item złożoność obliczeniowa;eksplozja stanów,
	\item wykonanie abstrakcji wymaga pracy eksperta.
\end{itemize}

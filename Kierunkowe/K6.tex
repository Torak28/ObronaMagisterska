\subsection{Problemy bezpieczeństwa transakcji zawieranych przy pomocy komunikacji bezprzewodowej}

W sieci bezprzewodowej występuje wiele problemów bezpieczeństwa - w przypadku sieci jesteśmy narażeni na m.in. sniffing, spoofing (SMTP, WWW, ARP, DNS itd.). Ze względu na łatwy dostęp do medium transmisyjnego, przez które następuje wymiana danych łatwiejsze są ataki Main in the Middle, Evil Twin, spoofing MAC, modyfikowanie pakietów, złamanie zabezpieczeń WEP, WPA, podsłuchwanie nieszyfrowanych transmisji.

\subsubsection{Man in the Middle}

Atak polega na przechwyceniu niezbędnych do wykonania transakcji danych i tym samym kradzieży środków lub tożsamości. W celu wykonania ataku należy przechwycić komunikację pomiędzy osobą atakowaną, a bankiem, następnie przekonać użytkownika, że strona, na którą się loguje (spreparowana przez nas) jest stroną, na którą chce się zalogować. Za pomocą otrzymanych danych można uzyskać dostęp od konta użytkownika. \\

Aby uchronić się przed tym atakiem należy sprawdzać certyfikaty stron podczas wykonywania transakcji, nie wykonywać transakcji będąc podłączonym do sieci publicznej. Banki, aby się przed tym bronić wprowadziły weryfikację SMS, w treści SMS możemy zweryfikować numer konta bankowego oraz kwotę przelewu.


\subsubsection{Sniffing transmisji}

Atak polega na podsłuchiwaniu transmisji nieszyfrowanych, dzięki czemu możliwe jest uzyskanie danyh takich jak login czy hasło do różnych systemów komputerowych, co pozwala na podszywanie się pod atakującą osobę. Aby dokonać ataku wystarczy oprogramowanie pozwalajace na to (np. Whireshark) lub odpowiendio skonfigurowana karta sieciowa. \\
Aby uchronić się przed tym atakiem można skonfigurować sieć, aby wymuszała szyfrowanie danych, nawiązywanie komunikacji w wyższych warstwach sieci w sposób szyfrowany np. TLS 1.1, SSL.

\subsubsection{Evil Twin}

Atak polega na podstawieniu urządzenia (np. WiFi Pineapple) , które symuluje Access Point. W przypadku, gdy urządzenie zacznie wygłaszać mocniejszy sygnał od prawdziwego nastąpi przełączenie bez informowania użytkownika. Dzięki temu cały ruch z komputera będzie przechodził przez fałszywy Access Point, co daje możliwość łatwego przechwycenia danych.

Będąc na konferencji niebezpiecznika pokazywali urządzenie WiFi Pineapple, którego używają do pentestów. Gdy w telefonie zostaje włączone WiFi, telefon wyszukuje dostępne wokół niego sieci oraz odpytuje je ”czy jesteś zapamiętaną przeze mnie siecią x?”, a urządzenie WiFi Pineapple zawsze odpowiada ”TAK”, dzięki czemu telefon się łączy z fake’owym Access Pointem.

W celu ochrony należy łączyć się tylko ze sprawdzonymi sieciami, wykorzystywać zabezpieczoną komunikację, oraz wyłączać WiFi w telefonie, gdy go nie używamy.

\subsubsection{Spoofing MAC}

Jedną z metod zabezpieczenia dostępów do sieci bezprzewodowej jest weryfikacja urządzeń po adresie fizycznym karty sieciowej MAC. Jednak podsłuchiwanie transmisji w sieciach bezprzewodowych jest stosunkowo proste, dlatego jest możliwość odczytania adresów MAC urządzeń, które mają dostęp do danych zasobów, a następnie podmienienie własnego adresu fizycznego MAC na jeden z nich.

Aby uchronić się przed tym atakiem należy oprócz weryfikacji adresów MAC używać zabezpieczeń hasłem lub nadawanie dostępu do zasobów dla użytkowników z dostępem do lokalnej sieci.

\subsubsection{Złamanie zabezpieczeń WEP i WPA}

Zabezpieczenia za pomocą WEP i WPA są mało bezpieczne, przy odpowiednim ataku możliwe jest zgadnięcie w ciągu kilkunastu/kilkudziesięciu minut (np. za pomocą narzędzia AirSnort). Dlatego należy używać lepszych zabezpieczeń - najskuteczniejszym obecnie szyfrowaniem jest WPA2, które używa algorytmu Advanced Encryption Standard (AES).


\subsubsection{Modyfikowanie pakietów}

W przypadku przechwycenia transmisji możliwe staje się zmodyfikowanie adresów, z którego została wysłana wiadomość, przez co możliwe staje się przechwycenie odpowiedzi serwera. Aby zabezpieczyć się przed modyfikowaniem pakietów konieczne jest zapewnienie integralności pakietów i jej weryfikacja.

\subsubsection{Przechwytywanie sygnału bluetooth}

Temat dotyczy komunikacji bezprzewodowej, więc również bluetooth.

Problem dotyczy klawiatur bezprzewodowych. Atakujący może dysponować urządzeniem, które będzie przechwytywało wciśnięte przez użytkownika przyciski, co po odpowiedniej analizie pozwoli na wyodrębnienie loginów i haseł.

Problemy z bezpieczeństwem RFID - można łatwo zczytać kartę, sklonować ją.

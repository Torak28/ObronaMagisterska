\subsection{Metody i narzędzia wykorzystywane w opisywaniu procesów biznesowych}

Proces biznesowy lub metoda biznesowa jest to seria powiązanych ze sobą działań lub zadań, które
rozwiązują określony problem lub prowadzą do osiągnięcia określonego efektu.

\subsubsection{Rodzaje procesów biznesowych}

Rodzaje procesów biznesowych:

\begin{itemize}
	\item \textbf{Proces zarządczy} – proces kierujący działaniem systemu (proces zarządzania przedsiębiorstwem)
	\item \textbf{Proces operacyjny} – stanowi istotę biznesu i jest źródłem wartości dodanej (zaopatrzenie, produkcja, marketing, sprzedaż)
	\item \textbf{Proces pomocniczy} – wspiera procesy główne (księgowość, rekrutacja, wsparcie techniczne) \\
\end{itemize}

W zależności od specyfiki organizacji i branży, wybrać można różne technologie pozwalające na zarządzanie procesami biznesowymi.

\subsubsection{Notacje opisu procesów biznesowych}

Wyróżnia się dwie najpopularniejsze notacje::

\begin{itemize}
	\item \textbf{BPMN} (\textit{Bussines Process Modeling Notation}) - jest to graficzna notacja służąca do przedstawiania i opisywania procesów biznesowych. Elementy graficzne i zasady ich używania przy tworzeniu diagramów są na tyle przejrzyste i nieskomplikowane, że zarówno specjaliści techniczni jak i przedstawiciele biznesu mogą odnaleźć dzięki niej wspólny język. 
	\item \textbf{UML} (\textit{Unified Modeling Language}) - język formalny do modelowania różnego rodzaju systemów oraz
	specyfikowania ich elementów. \\
\end{itemize}

\subsubsection{BPMN - Business Process Modelling Notation}

Graficzny język wizualizacji, specyfikowania, tworzenia i dokumentowania procesów biznesowych. Służy do opisu procesów na potrzeby systemów \textbf{ERP} (zarządzanie zasobami w przedsiębiorstwach) oraz \textbf{WorkFlow} (obieg pracy, dokumentów). Celem BPMN jest \textbf{wymiana procesów biznesowych na poziomie ludzi zamiast poziomu technicznego}. Wspiera proste i złożone procesy, oraz prostą wymianę komunikacji między użytkownikami - podstawą są diagramy procesów biznesowych. BPMN tworzy most między notacją pojęciową i techniczną - przekształca pomysł na kod (np. język BPL). Proces biznesowy w BPMN to zaplanowany proces (zbiór wzajemnie powiązanych działań, które przekształcają elementy wejściowe na wyjściowe) nastawiony na uzyskiwanie określonych rezultatów. \\

BPMN koncentruje się na sekwencji procesu z trzema rodzajami podprocesów:

\begin{itemize}
	\item \textbf{Prywatne} (Wewnętrzne procesy biznesowe) - Wykonywane są wewnątrz jednostki organizacyjnej (w ogólnych przepływach pracy procesów BPM),
	\item \textbf{Publiczne} (Abstrakcyjne procesy)- ilustrują interakcję między wewnętrznymi procesami a zewnętrznymi partnerami biznesowymi (wskazując sekwencję interakcji/wiadomości wewnętrznego procesu z zewnętrznymi użytkownikami). Ponadto przedstawiane są tylko czynności obejmujące zewnętrzną komunikację (wskazując przepływ sterowania),
	\item \textbf{Globalne} (Procesy współpracy) - ilustrują interakcje między dwoma lub więcej partnerami biznesowymi. Sekwencja czynności pokazuje wymianę wiadomości między jednostkami. \\
\end{itemize}

Wykorzystywana w BPMN służy jedynie do modelowania procesów biznesowych. Dzieli się ona na 4 kategorie:

\begin{itemize}
	\item \textbf{Elementy przepływu} - podstawa diagramu procesów biznesowych. Wśród nich wyróżnić
	można trzy podzbiory:
	\begin{itemize}
		\item \textbf{Zdarzenia} - prezentacja wydarzeń, które wystąpią lub mogą wystąpić w trakcie wykonywania procesu. Wpływają na przebieg procesu i ich wystąpienie jest czymś spowodowane lub powoduje skutek.
		\item \textbf{Bramki} - są elementami pozwalającymi na kontrolę przebiegu procesu, jego rozgałęzień i połączeń. Utożsamiać je można z elementami decyzyjnymi.
		\item \textbf{Czynności} - Czynność jest ogólnym pojęciem określającym pracę, którą uczestnik procesu wykonuje. 
	\end{itemize}
	\item \textbf{Połączenia}  - zawierają elementy pozwalające na zaprezentowanie związku pomiędzy elementami na diagramie, niezależnie czy jest to prezentacja przepływu, czy użycia danego elementu przez inny na diagramie. Wyróżnia się trzy podzbiory tej kategorii:
	\begin{itemize}
		\item Przepływy \textbf{sekwencji} - używane są do pokazania kolejności, w jakiej Czynności będą wykonywane w ramach procesu.
		\item Przepływy \textbf{komunikatów} - używane są do pokazania wymiany komunikatów pomiędzy odrębnymi uczestnikami procesu.
		\item \textbf{Asocjacje} - służą do dołączania dodatkowych informacji do Elementów Przepływu. Strzałka na końcu Asocjacji wskazuje kierunek powiązania.
	\end{itemize}
	\item Kategoria \textbf{Miejsca} realizacji zawiera elementy pozwalające na grupowanie obiektów procesu biznesowego zgodnie z ich przynależnością do osoby, roli bądź jednostki organizacyjnej.
	\item Kategoria \textbf{Artefakty} zawiera elementy pozwalające na zapewnienie dodatkowych informacji o modelowanym procesie. Nie są one bezpośrednio związane z przebiegiem procesu, lub przebiegiem informacji. Wyróżnia się trzy elementy tej kategorii:
	\begin{itemize}
		\item \textbf{Obiekty danych} mogą być dołączane do Przepływów, ale nie mają wpływu na ich przebieg. Mogą zawierać informacje o tym, czego dana Czynność wymaga, aby mogła zostać wykonana lub co dana Czynność produkuje.
		\item \textbf{Grupy} służą do łączenia elementów diagramu i prezentowania pewnego ich związku. Grupa nie ma wpływu na Przepływy pomiędzy Czynnościami.
		\item \textbf{Adnotacje} są sposobem pozwalającym modelującemu na dołączenie do elementów diagramu dodatkowych informacji dla jego odbiorcy.
\\
	\end{itemize}
\end{itemize}

\underline{\textbf{Microsoft Visio}}

\subsubsection{UML}

UML to notacja umożliwiająca zaprezentowanie systemu w sposób graficzny, w postaci diagramów. Modele zapisane w języku UML prezentują system od ogółu do szczegółu, umożliwiając oglądanie modelu systemu z wybraną w danym momencie szczegółowością. \\

Najważniejsze diagramy UML:

\begin{itemize}
	\item \textbf{Przypadków użycia} - diagram ten jest agregatem funkcji usług, które wykonuje system.
	\begin{itemize}
		\item \textbf{Przypadek użycia}
		\item \textbf{Aktor}
		\item \textbf{Relacja}
	\end{itemize}
	\item \textbf{Diagram klas} - obrazuje pewien zbiór klas, interfejsów i kooperacji oraz związki między nimi. Jest on grafem złożonym z wierzchołków (klas, interfejsów, kooperacji) i łuków (reprezentowanych przez relacje). Diagram klas stanowi opis statyki systemu, który uwypukla związki między klasami, pomijając pozostałe charakterystyki. 
	\item \textbf{Diagram sekwencji} służy do prezentowania interakcji pomiędzy obiektami wraz z uwzględnieniem w czasie komunikatów, jakie są przesyłane pomiędzy nimi. 
	\begin{itemize}
		\item \textbf{Uczestnik/klasa/obiekt}
		\item \textbf{Linia życia}
		\item \textbf{Komunikat} \\
	\end{itemize}
\end{itemize}

\underline{\textbf{UMLet}}
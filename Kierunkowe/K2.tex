\subsection{Mechanizmy ochrony danych w systemach operacyjnych}

Dzisiejsze systemy są wieloużytkownikowe, co powoduje dodatkowe zagrożenie danych. System musi zabezpieczyć dane jednego użytkownika przed niepożądanym działaniem innego użytkownika, np. odczyt przez nieuprawnione osoby, nadpisanie lub usunięcie danych. Bezpieczny system musi zapewniać poufność, integralność oraz dostępność źródeł.

\subsubsection{Wirtualizacja pamięci}

Ponieważ podstawowym zadaniem systemu operacyjnego jest zapewnienie możliwości (pseudo)- równoczesnego wykonywania wielu procesów użytkownika – czyli uruchomionych w systemie programów – istotne jest zabezpieczenie przed wzajemnym czytaniem lub nadpisywaniem pamięci systemowej (RAM) przez różne procesy. We współczesnych systemach ten rodzaj ochrony jest zapewniany przez \textbf{wirtualizację pamięci} (chociaż nie jest to jedyny jej cel).\\

W systemie ze stronicowaną pamięcią wirtualną, każdy proces posiada swoją własną przestrzeń adresową, której rozmiar zależy od długości słowa maszynowego (w przypadku procesorów i systemów 64-bitowych – 264 bajtów) i generalnie może być znacznie większy od rozmiaru pamięci zainstalowanej fizycznie. Stosowane w takiej wirtualnej przestrzeni adresowej przez programy \textbf{adresy logiczne} są tłumaczone przez system operacyjny na \textbf{adresy fizyczne} dzięki \textbf{tablicom stron}, przechowującym informację o przyporządkowaniu określonych fragmentów przestrzeni adresowej procesu (stron) do odpowiednich fragmentów pamięci fizycznej (ramek). Aby utrudnić ataki na oprogramowanie za pomocą exploitów, dodatkowo może zostać zastosowana \textbf{randomizacja przestrzeni adresowej} (ang. \textit{\textbf{ASLR} – Adress Space Layout Randomization}). Polega ona na losowym rozmieszczaniu kluczowych obszarów (np. segmentu danych i segmentu stosu) w wirtualnej przestrzeni adresowej procesu, dzięki czemu atakującemu trudniej jest przewidzieć, w którym miejscu powinien podłożyć potrzebne do dokonania ataku dane (np. spreparowany adres powrotny z funkcji). \\

\subsubsection{Prawa dostępu}

Innym zadaniem realizowanym przez system operacyjny jest ochrona danych przechowywanych w pamięci masowej (na dysku) przed dostępem przez różnych użytkowników. Służą do tego uprawnienia na poziomie systemu plików:
\begin{itemize}
	\item  \textbf{w systemie Windows z systemem plików NTFS} – każdy plik ma przypisanego właściciela (użytkownika albo grupę), istnieje pięć podstawowych rodzajów uprawnień (pełna kontrola, modyfikacja, wykonanie, odczyt, zapis) i wiele rodzajów specjalnych (np. przejęcie na własność), które przydziela się dla konkretnych użytkowników lub grup za pomocą list ACL, oprócz tego występuje dość skomplikowane dziedziczenie uprawnień w hierarchii katalogów,
	\item  \textbf{w systemach uniksowych (m. in. Linux)} – każdy plik ma przypisanego właściciela zwykłego (konto użytkownika) i właściciela grupowego (grupę użytkowników) oraz istnieją, nie licząc kilku specjalnych, trzy rodzaje uprawnień – \textbf{do odczytu} (\textit{r}), \textbf{zapisu} (\textit{w}) i \textbf{wykonania} (\textit{x}) – które podstawowo przydziela się osobno dla \textbf{właściciela}(\textit{u}), \textbf{właściciela grupowego}(\textit{g}) i \textbf{wszystkich pozostałych użytkowników}(\textit{o}), a jeżeli taki podział nie jest wystarczający – można wykorzystać listy ACL, które pozwalają te same uprawnienia przydzielać dowolnym innym użytkownikom lub grupom,
\end{itemize}

\subsection{Definicje, charakterystyka i zastosowania rzeczywistości rozszerzonej i wirtualnej}

\subsubsection{Rzeczywistość rozszerzona(\textit{Augmented reality})}

System łączący świat rzeczywisty, realny, oraz rzeczywistość wirtualna. Taki system jest interaktywny w czasie rzeczywistym, umożliwia on swobodę ruchów w trzech wymiarach. W celu wytworzenia rzeczywistości rozszerzonej zwykle jest potrzebny aparat, ekran na którym będzie wyświetlana informacja oraz generowana w czasie rzeczywistym grafika 3D lub 2D. AR można podzielić na trzy rodzaje: 

\begin{itemize}
	\item AR bazujący się na \textbf{znaczniku} lub \textbf{znacznikach} 
	\item AR który \textbf{nie używa znaczników }
	\item AR bazujący się na \textbf{lokalizacji}. \\
\end{itemize}

Pierwszy rodzaj AR używa tak zwane \textbf{znaczniki}. Znacznik to obraz lub obiekt który jest elementem wskazującym na miejsce gdzie musi być umieszczony trójwymiarowy obiekt, jego rozmiar i położenie. Plusem wykorzystania znaczników można nazwać wysoką precyzje położenia wygenerowanych obiektów 3D lub 2D. Minusem znaczników można nazwać to że gdy on zostanie zgubiony, cała scena AR znika. \\

AR który \textbf{nie używa znaczników} zazwyczaj \textbf{skanuje otoczenia} za pomocą kamery w celu wykrycia płaskiej powierzchni, na której potem będzie umieszczony obiekt 3D. Zaletą takiego typu AR jest to że gdy taka powierzchnia zostanie rozpoznana, użytkownik nie straszne stracenie znacznika AR i może łatwiej poruszać wokół zeskanowanej powierzchni. Jednak za taką swobodę użytkownik płaci czasem potrzebnym na zeskanowanie tej powierzchni. \\
 
Trzeci rodzaj AR bazuje się na lokalizacji. Scena AR zostanie uruchomiona w \textbf{odpowiednim miejscu}, dla którego ona \textbf{została zaprojektowana}. \\

Istnieje wiele różnych rodzajów urządzeń za pomocą których AR może być wyświetlany, między innymi to:

\begin{itemize}
	\item \textbf{Handheld AR} ( do tego można odnieść smartphone i tablet), 
	\item \textbf{Smart glasses} ( optyczne okulary, czyli takie w których używane przezroczyste lub pół przezroczyste powierzchnie do wyświetlania grafiki 3D lub 2D, \textbf{Microsoft’s Hololens}, Magic Leap One, Google Glass. Okulary VR posiadające zintegrowane aparaty, w takich okularach na ekran lub ekrany przekazywany jest widok otoczenia z aparatów oraz wygenerowane grafiki 3D lub 2D, HTC Vive VR ) 
	\item \textbf{HUD} - Head-Up Display ( przezroczysta lub pół przezroczysta powierzchnia na którą wyświetlana jest projekcja, projektor lub ekran który przekazuje informacje na powierzchnię oraz komputer generujący obrazy do wyświetlania ) 
	\item Ekrany montowane na kasku ( działa podobnie do HUD ) 
	\item Ekrany holograficzne ( Istnieje wiele różnych rodzajów takich typów ekranów, prostym przykładem jest stworzenie tak zwanej piramidy, umieszczenia szkła pod kątem 45 stopni )
\end{itemize}

\subsubsection{Zastosowanie AR}

\begin{itemize}
	\item \textbf{AR w przemyśle}
	
	Używanie okularów AR dla wspomagania pracownika w celu wytwarzania jakiegoś produktu lub sprawdzenia jakości jego wytworzenia ( zastosowanie \textbf{Google Glass w fabrykach Boeing}, BIManywhere i TrimbleConnect dla budownictwa — pozwala pokazać rury i kable które muszą być zainstalowane w budynku, sprawdzenie czy wszystko zostało zrealizowane zgodnie z projektem ) 
	
	\item \textbf{AR w medycynie: }
	
	Demonstracja różnych chorób dla pacjenta czy dla uczenia uczniów i lekarzy ( Demonstracji anatomii dla uczniów, demonstracji skutków choroby dla pacjenta — jeżeli pacjent ma chore wzroku , można jemu zademonstrować skutki nie leczenia tej choroby ) 
	
	\item \textbf{AR dla wojska: }
	
	Informowanie żołnierzy o swoim otoczeniu, informacja dla pilota
\end{itemize}

\subsubsection{Rzeczywistość wirtualna (\textit{virtual reality})}

Komputerowo wygenerowany trójwymiarowy obraz, który \textbf{imituje świat realny} lub\textbf{ stanowi wizję świata fikcyjnego}. W takie wygenerowane przez komputer otoczenie osoba może całkowicie się zanurzyć. Taki system tak samo jest systemem interaktywnym i umożliwia ruch w trzech wymiarach za pomocą dedykowanych sensorów lub kamer. W celu umieszczenia osoby w rzeczywistości wirtualnej zwykle jest potrzebny ekran lub dwa ekrany, specjalne soczewki oraz komputer generujący ten wirtualny świat. Okulary VR mogą być dwóch rodzajów: wyposażone lub nie wyposażone w specjalny mini komputer generujący ten świat. \\

Przykładem zwykłych okularów VR są \textbf{Oculus Rift}, \textbf{HTC Vive} i inne. Oculus Quest z innej strony zawiera w sobie płytę główną, procesor i baterie dla pracy autonomicznej, bez podłączania do komputera. Istnieje również możliwość umieszczenia smartfona w dedykowanych okularach. Interakcja z takim wirtualnym światem może odbywać za pomocą zwykłego kontrolera lub nawet klawiatury i myszy. \\

Zaawansowane systemy VR wyposażone w specjalne kontrolery ułatwiające sterowanie otoczeniem. \\

Systemy VR również mogą zawierać sensory które mapują położenie użytkownika w świecie fizycznym przenosząc dane o jego położeniu do świata wirtualnego. Takie sensory najczęściej umieszczane na suficie aby pokrywać dużą powierzchnie. Dodatkowo można umieścić sensory na czele użytkownika aby lepiej przenieść jego ruch do świata wirtualnego 

\subsubsection{Zastosowanie VR}

\begin{itemize}
	\item \textbf{Diagnozowanie i leczenie chorób} związanych z zdrowiem psychologicznym np. rehabilitacja osób z Alzheimerem
	\item Używanie \textbf{VR jako terapii} ( Za pomocą VR lekarzy mogą wizualizować halucynacji dla osób chorych na Schizofrenię co pomaga walczyć z nimi, też jest możliwe przeprowadzenie terapii przeciwbólowej ) 
	\item \textbf{Trenowanie i uczenie} (na przykład trenowanie lekarzy dla przeprowadzenia skomplikowanych operacji, trenowanie wojskowych)
\end{itemize}
